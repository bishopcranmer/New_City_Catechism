\documentclass[00-main.tex]{subfiles}

\begin{document}

\mainmatter

\part{\em New City Catechism}

\chapter[Part 1. God, Creation And Fall, Law][God, Creation And Fall, Law]{Part 1. God, Creation And Fall, Law}

\section{Q  1. What is our only hope in life and death?}
\Children{That we are not our own but belong, body and soul, both in life and death, to God and to our Savior Jesus Christ.}

\Quote{Romans 14:7--8\index[rom]{14:7--8}}{For none of us lives to himself, and none of us dies to himself. For if we live, we live to the Lord, and if we die, we die to the Lord. So then, whether we live or whether we die, we are the Lord's.}

\subsection{Commentary}
\begin{quote}
``If we, then, are not our own but the Lord's, it is clear what error we must flee, and whither we must direct all the acts of our life. We are not our own: let not our reason nor our will, therefore, sway our plans and deeds. We are not our own: let us therefore not set it as our goal to seek what is expedient for us \dots. We are not our own: in so far as we can, let us forget ourselves and all that is ours. Conversely, we are God's: let us therefore live for him and die for him. We are God's: let his wisdom and will therefore rule all our actions. We are God's: let all the parts of our life accordingly strive toward him as our only lawful goal. O, how much has that man profited who, having been taught that he is not his own, has taken away dominion and rule from his own reason that he may yield it to God! For, as consulting our self-interest is the pestilence that most effectively leads to our destruction, so the sole haven of salvation is to be wise in nothing and to will nothing through ourselves but to follow the leading of the Lord alone.''\end{quote}

\begin{flushright}
John Calvin\index[people]{Calvin, John} \cite{Calvin:1960}, III. VII. I., 690.
\end{flushright}

\subsection{Prayer}

\begin{quote}
	``Lord, here am I; do with me what thou pleasest, write upon me as thou pleasest: I give up myself to be at thy dispose \dots. The ambitious man giveth himself up to his honours, but I give up myself unto thee; \dots\ man gives himself up to his pleasures, but I give up myself to thee; \dots\ man gives himself up \dots\ to his idols, but I give myself to thee \dots\. Lord! lay what burden thou wilt upon me, only let thy everlasting arms be under me \dots. I am lain down in thy will, I have learned to say amen to thy amen; thou hast a greater interest in me than I have in myself, and therefore I give up myself unto thee, and am willing to be at thy dispose, and am ready to receive what impression thou shalt stamp upon me. O blessed Lord! hast thou not again and again said unto me \dots\ `I am thine, O soul! to save thee; my mercy is thine to pardon thee; my blood is thine to cleanse thee; my merits are thine to justify thee; my righteousness is thine to clothe thee; my Spirit is thine to lead thee; my grace is thine to enrich thee; and my glory is thine to reward thee'; and therefore \dots\ I cannot but make a resignation of myself unto thee. Lord! here I am, do with me as seemeth good in thine own eyes. I know the best way \dots\ is to resign up myself to thy will, and to say amen to thy amen.
\end{quote}

\begin{flushright}
	Thomas Brooks \cite{Brooks:1866}, 305--306.\index[people]{Brooks, Thomas}
\end{flushright}

\begin{paracol}{2}
	
\subsection{Further Reading}

``Salvation'' in \cite{Packer:2001}.\index[people]{Packer, J.I.}

\switchcolumn

\subsection{Video}

\href{https://vimeo.com/46824533}{vimeo.com\Slash{}46824533}

\end{paracol}


\section{Q 2. What Is God?}
\Children{God is the creator }and sustainer \Children{of everyone and everything.} He is eternal, infinite, and unchangeable in his power and perfection, goodness and glory, wisdom, justice, and truth. Nothing happens except through him and by his will.

\Quote{Psalm 86:8--10\index[psa]{86:8--10} and 15\index[psa]{86:15}}{Among the gods there is none like you, Lord; no deeds can compare with yours. All the nations you have made will come and worship before you, Lord; they will bring glory to your name. For you are great and do marvelous deeds; you alone are God \dots You, Lord, are a compassionate and gracious God, slow to anger, abounding in love and faithfulness.}

\subsection{Commentary}
\begin{quote}
``God is an eternal, independent being \dots. He gives being to all creatures \dots. God is an eternal, unchangeable being \dots. His being is without any limits. Angels and men have their beings, but then they are bounded and limited; \dots but God is an immense being that cannot be included within any bounds \dots. There never was nor shall be time wherein God could not say of himself, `I am' \dots. He is a God that gives being to all things \dots. He is the Being of beings, subsisting by himself; \dots `I am that I am, and as I am, so will I be to all eternity' \dots. He is infinite in power, sovereign in dominion, and not bounded as creatures are \dots. He is so strong that he is almighty, he is one to whom nothing is impossible \dots. He wanteth nothing, but is infinitely blessed with the infinite perfection of his glorious being \dots self-sufficient, all-sufficient, absolutely perfect \dots. There is no succession or variation in God, but he is eternally the same \dots. God ever was, ever is, and ever shall be. Though the manifestations of himself unto the creatures are in time, yet his essence or being never did nor shall be bound up by time. Look backward or forward, God from eternity to eternity, is a most self-sufficient, infinite, perfect, blessed being, the first cause of our being, and without any cause of his own being; an eternal infinite fulness, and possession to himself and of himself. What God is, he was from eternity, and what God is, he will be so to eternity.''
\end{quote}

\begin{flushright}
Thomas Brooks \cite{Brooks:1866a}, 150--157.\index[people]{Brooks, Thomas}
\end{flushright}

\subsection{Prayer}
\begin{quote}
``I believe, O sovereign Goodness, O mighty Wisdom, that thou dost sweetly order and govern all things, even the most minute, even the most noxious, to thy glory, and the good of those that love thee. I believe, O Father of the families of heaven and earth, that thou so disposest all events, as may best magnify thy goodness to all thy children, especially those whose eyes wait upon thee. I most humbly beseech thee, teach me to adore all thy ways, though I cannot comprehend them; teach me to be glad that thou art king, and to give thee thanks for all things that befall me; seeing thou hast chosen that for me, and hast thereby `set to thy seal that they are good'. And for that which is to come, give me thy grace to do in all things what pleaseth thee; and then, with an absolute submission to thy wisdom, to leave the issues of them in thy hand.''
\end{quote}

\begin{flushright}
John Wesley \cite{Wesley:1831}, 392.\index[people]{Wesley, John}
\end{flushright}



\begin{paracol}{2}
\subsection{Further Reading}

``Self-existence'', ``Transcendence'', and ``Almightiness'' in \cite{Packer:2001}.\index[people]{Packer, J.I.}

\switchcolumn 

\subsection{Video}
\href{https://vimeo.com/46824534}{vimeo.com\Slash{}46824534}
\end{paracol}

\section{Q  3. How many persons are there in God?}
\Children{There are three persons in }the \Children{one }true and living \Children{God: the Father, the Son, and the Holy Spirit.} They are the same in substance, equal in power and glory.

\Quote{2 Corinthians 13:14\index[1co]{13:14}}{May the grace of the Lord Jesus Christ, and the love of God, and the fellowship of the Holy Spirit be with you all.}

\subsection{Commentary}

\begin{quotation}
The \ldots Trinity, Father, Son, and Holy Ghost, being one God, is \ldots necessary to us to be believed, not only as to the eternal \ldots but especially for the knowledge of God's three great sorts of works on man: that is, as our Creator, and the God of nature; as our Redeemer, and the God of governing and reconciling grace, and as our Sanctifier, and the Applier and Perfecter of all to fit us to glory.

The Scripture tells us that there are three, and yet but one God. \ldots  We are to be baptised into the name of the Father, Son, and Holy Ghost (Matt. xxviii. 29.)\index[mat]{28:29} And there are three that bear record in heaven, the Father, the Word, and the Holy Spirit, and these three are one (1 John v. 7.)\index[1jo]{5:07@5:7} \ldots  [That] God is one infinite, undivided Spirit; and yet that he is Father, Son, and Holy Ghost, must be believed.

We must \ldots know, believe and esteem him to be the only infinite, eternal, self-sufficient Spirit, vital Power, Understanding, and Will, our most perfect Life, Light, and Love; Father, Son, and Holy Ghost, of whom, and through whom, and to whom, are all things; our absolute Owner, Ruler, and Father; our Maker, our Redeemer, and Sanctifier.
\end{quotation}

\begin{flushright}
	Richard Baxter\index[people]{Baxter, Richard} \cite{Baxter:1830}, 33, 62, 165. \label{baxter:q3}
\end{flushright}

\subsection{Prayer}

\begin{quotation}
Not without trembling, we have entreated of the most holy mystery of the reverend Trinity, the Father, the Son, and the Holy Ghost, which we have learned out of the scriptures: and here now we will stay, humbly worshipping this Unity in trinity and Trinity in unity. And let us keep in mind and acknowledge this distinction or division most manifestly declared in the scriptures, and the unity also commended unto us with exceeding great diligence \ldots. There is but one God \ldots. Therefore when we read that God created the world, we understand that the Father from whom are all things, by the Son by whom are all things, in the Holy Ghost in whom are all things, created the world. And when we read that the Son became flesh, suffered, died, and rose again for our salvation, we believe that the Father and the Holy Ghost, though they were not partakers of his incarnation and passion, yet notwithstanding that they wrought our salvation by the Son \ldots. And when sins are said to be forgiven in the Holy Ghost, we believe that this benefit and all other benefits of our blessedness are unseparably given and bestowed upon us from one, only, true, living, and everlasting God, who is the Father, the Son, and the Holy Ghost. To whom be praise and thanksgiving for ever and ever. Amen.
\end{quotation}

\begin{flushright}
	Henry Bullinger\index[people]{{Bullinger, Henry}|see {Bullinger, Heinrich}}\index[people]{Bullinger, Heinrich} \cite{Bullinger:1851}, 325--326
\end{flushright}






\begin{paracol}{2}
	\subsection{Further Reading}
	
	``Trinity'' in \cite{Packer:2001}.\index[people]{Packer, J.I.}
	
	\switchcolumn 
	
	\subsection{Video}
	\href{https://vimeo.com/46824536}{vimeo.com\slash{}46824536}
\end{paracol}




\section{Q  4. How and why did God create us?}
\Children{God created us male and female in his own image to }know him, love him, live with him, and \Children{glorify him.} And it is right that we who were created by God should live to his glory.

\Quote{Genesis 1:27\index[gen]{1:27}}{So God created mankind in his own image, in the image of God he created them; male and female he created them.}


\section{Q  5. What else did God create?}
\Children{God created all things }by his powerful Word, \Children{and all his creation was very good}; everything flourished under his loving rule\Children{.}

\Quote{Genesis 1:31\index[gen]{1:31}}{God saw all that he had made, and it was very good.}


\section{Q  6. How can we glorify God?}
We glorify God \Children{by }enjoying him, \Children{loving him}, trusting him, \Children{ and by obeying his }will, \Children{commands, and law.}

\Quote{Deuteronomy 11:1\index[deu]{11:1}}{Love the \textsc{Lord} your God and keep his requirements, his decrees, his laws and his commands always.}


\section{Q 7. What does the law of God require?}
Personal, perfect, and perpetual obedience; \Children{that we love God with all our heart, soul, mind, and strength; and love our neighbor as ourselves.} What God forbids should never be done and what God commands should always be done.

\Quote{Matthew 22:37--40\index[mat]{22:37--40}}{Jesus replied: ``Love the Lord your God with all your heart and with all your soul and with all your mind. This is the first and greatest commandment. And the second is like it: Love your neighbor as yourself. All the Law and the Prophets hang on these two commandments.''}

\section{Q 8. What is the law of God stated in the Ten Commandments?}
\Children{You shall have no other gods before me. You shall not make for yourself an idol} in the form of anything in heaven above or on the earth beneath or in the waters below\thinspace{}---\thinspace{}you shall not bow down to them or worship them\Children{. You shall not misuse the name of the \textsc{Lord} your God. Remember the Sabbath day by keeping it holy. Honor your father and your mother. You shall not murder. You shall not commit adultery. You shall not steal. You shall not give false testimony. You shall not covet.}

\Quote{Exodus 20:3\index[exo]{20:3} and Deuteronomy 5:7\index[deu]{05:7@5:7}}{You shall have no other gods before me.}

\section{Q 9. What does God require in the first, second, and third commandments?}
\Children{First, that we know }and trust \Children{God as the only true }and living \Children{God. Second, that we avoid all idolatry }and do not worship God improperly\Children{. Third, that we treat God's name with fear and reverence}, honoring also his Word and works\Children{.}

\Quote{Deuteronomy 6:13--14\index[deu]{06:13--14@6:13--14}}{Fear the \textsc{Lord} your God, serve him only and take your oaths in his name. Do not follow other gods, the gods of the peoples around you.}

\section{Q 10. What does God require in the fourth and fifth commandments?}
\Children{Fourth, that on the Sabbath day we spend time in }public and private \Children{worship of God}, rest from routine employment, serve the Lord and others, and so anticipate the eternal Sabbath\Children{. Fifth, that we love and honor our father and our mother}, submitting to their godly discipline and direction\Children{.}

\Quote{Leviticus 19:3\index[lev]{19:3}}{Each of you must respect your mother and father, and you must observe my Sabbaths. I am the \textsc{Lord} your God.}

\section{Q 11. What does God require in the sixth, seventh, and eighth commandments?}
\Children{Sixth, that we do not hurt, or hate}, or be hostile to \Children{ our neighbor}, but be patient and peaceful, pursuing even our enemies with love\Children{. Seventh, that we }abstain from sexual immorality and \Children{live purely and faithfully}, whether in marriage or in single life, avoiding all impure actions, looks, words, thoughts, or desires, and whatever might lead to them\Children{. Eighth, that we do not take without permission that which belongs to someone else}, nor withhold any good from someone we might benefit\Children{.}

\Quote{Romans 13:9\index[rom]{13:9}}{The commandments, ``You shall not commit adultery,'' ``You shall not murder,'' ``You shall not steal,'' ``You shall not covet,'' and whatever other command there may be, are summed up in this one command: ``Love your neighbor as yourself.''}

\section{Q 12. What does God require in the ninth and tenth commandments?}
\Children{Ninth, that we do not lie or deceive}, but speak the truth in love\Children{. Tenth, that we are content, not envying anyone} or resenting what God has given them or us\Children{.}

\Quote{James 2:8\index[jam]{2:8}}{If you really keep the royal law found in Scripture, ``Love your neighbor as yourself,'' you are doing right.}

\section{Q 13. Can anyone keep the law of God perfectly?}
\Children{Since the fall, no }mere \Children{human has been able to keep the law of God perfectly}, but consistently breaks it in thought, word, and deed\Children{.}

\Quote{Romans 3:10--12\index[rom]{03:10--12@3:10--12}}{There is no one righteous, not even one; there is no one who understands; there is no one who seeks God. All have turned away, they have together become worthless; there is no one who does good, not even one.}

\section{Q 14. Did God create us unable to keep his law?}
\Children{No, but because of the disobedience of }our first parents, \Children{Adam and Eve}, all of creation is fallen\Children{; we are all born in sin and guilt,} corrupt in our nature and \Children{ unable to keep God's law.}

\Quote{Romans 5:12\index[rom]{05:12@5:12}}{Therefore, just as sin entered the world through one man, and death through sin, and in this way death came to all people, because all sinned.}

\section{Q 15. Since no one can keep the law, what is its purpose?}
\Children{That we may know the holy nature }and will \Children{of God, and the sinful nature }and disobedience \Children{of our hearts; and thus our need of a Savior.} The law also teaches and exhorts us to live a life worthy of our Savior.

\Quote{Romans 3:20\index[rom]{03:20@3:20}}{No one will be declared righteous in God's sight by the works of the law; rather, through the law we become conscious of our sin.}

\section{Q 16. What is sin?}
\Children{Sin is rejecting or ignoring God in the world he created, }rebelling against him by living without reference to him, \Children{not being or doing what he requires in his law}\thinspace{}---\thinspace{}resulting in our death and the disintegration of all creation\Children{.}

\Quote{1 John 3:4\index[1jo]{3:4}}{Everyone who sins breaks the law; in fact, sin is lawlessness.}

\section{Q 17. What is idolatry?}
\Children{Idolatry is trusting in created things rather than the Creator} for our hope and happiness, significance and security\Children{.}

\Quote{Romans 1:21\index[rom]{01:21@1:21} and 25\index[rom]{01:25@1:25}}{For although they knew God, they neither glorified him as God nor gave thanks to him, but their thinking became futile and their foolish hearts were darkened\,\dots{}\,They exchanged the truth about God for a lie, and worshiped and served created things rather than the Creator.}

\section{Q 18. Will God allow our disobedience and idolatry to go unpunished?}
\Children{No, }every sin is against the sovereignty, holiness, and goodness of God, and against his righteous law, and \Children{God is righteously angry with our sins and will punish them }in his just judgment \Children{both in this life, and in the life to come.}

\Quote{Ephesians 5:5--6\index[eph]{5:5--6}}{For of this you can be sure: No immoral, impure or greedy person\thinspace{}---\thinspace{}such a person is an idolater\thinspace{}---\thinspace{}has any inheritance in the kingdom of Christ and of God. Let no one deceive you with empty words, for because of such things God's wrath comes on those who are disobedient.}

\section{Q 19. Is there any way to escape punishment and be brought back into God's favor?}
\Children{Yes, }to satisfy his justice, \Children{God} himself, out of mere mercy, \Children{reconciles us to himself }and delivers us from sin and from the punishment for sin, \Children{by a Redeemer.}

\Quote{Isaiah 53:10--11\index[isa]{53:10--11}}{Yet it was the \textsc{Lord's} will to crush him and cause him to suffer, and though the \textsc{Lord} makes his life an offering for sin, he will see his offspring and prolong his days, and the will of the \textsc{Lord} will prosper in his hand. After he has suffered, he will see the light of life and be satisfied; by his knowledge my righteous servant will justify many, and he will bear their iniquities.}

\section[Q 20. Who is the Redeemer?]{Q 20. Who is the Redeemer?}
\Children{The only Redeemer is the Lord Jesus Christ}, the eternal Son of God, in whom God became man and bore the penalty for sin himself\Children{.}

\Quote{1 Timothy 2:5\index[1ti]{2:5}}{For there is one God and one mediator between God and mankind, the man Christ Jesus.}


\chapter[Part 2. Christ, Redemption, Grace][Christ, Redemption, Grace]{Part 2. Christ, Redemption, Grace}

\section{Q 21. What sort of Redeemer is needed to bring us back to God?}
\Children{One who is truly human and also truly God.}

\Quote{Isaiah 9:6\index[isa]{09:6@9:6}}{For to us a child is born, to us a son is given, and the government will be on his shoulders. And he will be called Wonderful Counselor, Mighty God, Everlasting Father, Prince of Peace.}

\section{Q 22. Why must the Redeemer be truly human?}
\Children{That in human nature he might on our behalf perfectly obey the whole law and suffer the punishment for human sin}; and also that he might sympathize with our weaknesses\Children{.}

\Quote{Hebrews 2:17\index[heb]{2:17}}{For this reason he had to be made like them, fully human in every way, in order that he might become a merciful and faithful high priest in service to God, and that he might make atonement for the sins of the people.}

\section{Q 23. Why must the Redeemer be truly God?}
\Children{That because of his divine nature his obedience and suffering would be perfect and effective}; and also that he would be able to bear the righteous anger of God against sin and yet overcome death\Children{.}

\Quote{Acts 2:24\index[act]{02:24@2:24}}{But God raised him from the dead, freeing him from the agony of death, because it was impossible for death to keep its hold of him.}

\section{Q 24. Why was it necessary for Christ, the Redeemer, to die?}
Since death is the punishment for sin, \Children{Christ died willingly in our place to deliver us from the power and penalty of sin and bring us back to God.} By his substitutionary atoning death, he alone redeems us from hell and gains for us forgiveness of sin, righteousness, and everlasting life.

\Quote{Colossians 1:21--22\index[col]{1:21--22}}{\sloppy Once you were alienated from God and were enemies in your minds because of your evil behavior. But now he has reconciled you by Christ's physical body through death to present you holy in his sight, without blemish and free from accusation.}

\section{Q 25. Does Christ's death mean all our sins can be forgiven?}
\Children{Yes, because Christ's death on the cross fully paid the penalty for our sin, God} graciously imputes Christ's righteousness to us as if it were our own and \Children{will remember our sins no more.}

\Quote{2 Corinthians 5:21\index[2co]{5:21}}{God made him who had no sin to be sin for us, so that in him we might become the righteousness of God.}

\section{Q 26. What else does Christ's death redeem?}
Christ's death is the beginning of the redemption and renewal of \Children{every part of fallen creation}, as he powerfully directs all things for his own glory and creation's good\Children{.}

\Quote{Colossians 1:19--20\index[col]{1:19--20}}{For God was pleased to have all his fullness dwell in him, and through him to reconcile to himself all things, whether things on earth or things in heaven, by making peace through his blood, shed on the cross.}

\section{Q 27. Are all people, just as they were lost through Adam, saved through Christ?}
\Children{No, only those who are elected by God and united to Christ by faith.} Nevertheless God in his mercy demonstrates common grace even to those who are not elect, by restraining the effects of sin and enabling works of culture for human well-being.

\Quote{Romans 5:17\index[rom]{05:17@5:17}}{For if, by the trespass of the one man, death reigned through that one man, how much more will those who receive God's abundant provision of grace and of the gift of righteousness reign in life through the one man, Jesus Christ!}

\section{Q 28. What happens after death to those not united to Christ by faith?}
At the day of judgment they will receive the fearful but just sentence of condemnation pronounced against them. \Children{They will be cast out from the} favorable \Children{presence of God, into hell, to be justly} and grievously \Children{punished, forever.}

\Quote{John 3:16--18\index[joh]{03:16--18@3:16--18} and 36\index[joh]{03:36@3:36}}{For God so loved the world that he gave his one and only Son, that whoever believes in him shall not perish but have eternal life. For God did not send his Son into the world to condemn the world, but to save the world through him. Whoever believes in him is not condemned, but whoever does not believe stands condemned already because they have not believed in the name of God's one and only Son\,\dots{}\,Whoever believes in the Son has eternal life, but whoever rejects the Son will not see life, for God's wrath remains on them.}

\section{Q 29. How can we be saved?}
\Children{Only by faith in Jesus Christ and in his substitutionary atoning death on the cross}; so even though we are guilty of having disobeyed God and are still inclined to all evil, nevertheless, God, without any merit of our own but only by pure grace, imputes to us the perfect righteousness of Christ when we repent and believe in him\Children{.}

\Quote{Ephesians 2:8--9\index[eph]{2:8--9}}{For it is by grace you have been saved, through faith\thinspace{}---\thinspace{}and this is not from yourselves, it is the gift of God\thinspace{}---\thinspace{}not by works, so that no one can boast.}

\section{Q 30. What is faith in Jesus Christ?}
Faith in Jesus Christ is acknowledging the truth of everything that God has revealed in his Word, trusting in him, and also \Children{receiving and resting on him alone for salvation as he is offered to us in the gospel.}

\Quote{Galatians 2:20\index[gal]{2:20}}{I have been crucified with Christ and I no longer live, but Christ lives in me. The life I now live in the body, I live by faith in the Son of God, who loved me and gave himself for me.}

\section{Q 31. What do we believe by true faith?}
Everything taught to us in the gospel. The Apostles' Creed expresses what we believe in these words: \Children{We believe in God the Father Almighty, Maker of heaven and earth; and in Jesus Christ his only Son our Lord, who was conceived by the Holy Spirit, born of the virgin Mary, suffered under Pontius Pilate, was crucified, died, and was buried. He descended into hell. The third day he rose again from the dead. He ascended into heaven, and is seated at the right hand of God the Father Almighty; from there he will come to judge the living and the dead. We believe in the Holy Spirit, the holy catholic church, the communion of saints, the forgiveness of sins, the resurrection of the body, and the life everlasting.}

\Quote{Jude 1:3\index[jud]{1:3}}{I\,\dots{}\,urge you to contend for the faith that was once for all entrusted to God's holy people.}

\section{Q 32. What do justification and sanctification mean?}
\Children{Justification means our declared righteousness before God}, made possible by Christ's death and resurrection for us\Children{.} \Children{Sanctification means our gradual, growing righteousness}, made possible by the Spirit's work in us\Children{.}

\Quote{1 Peter 1:1--2\index[1pe]{1:1--2}}{To God's elect\,\dots{}\,who have been chosen according to the foreknowledge of God the Father, through the sanctifying work of the Spirit, to be obedient to Jesus Christ and sprinkled with his blood: Grace and peace be yours in abundance.}

\section{Q 33. Should those who have faith in Christ seek their salvation through their own works, or anywhere else?}
\Children{No, }they should not, \Children{as everything necessary to salvation is found in Christ.} To seek salvation through good works is a denial that Christ is the only Redeemer and Savior.

\Quote{Galatians 2:16\index[gal]{2:16}}{Know that a person is not justified by the works of the law, but by faith in Jesus Christ. So we, too, have put our faith in Christ Jesus that we may be justified by faith in Christ and not by the works of the law, because by the works of the law no one will be justified.}

\section{Q 34. Since we are redeemed by grace alone, through Christ alone, must we still do good works and obey God's Word?}
\Children{Yes, }because Christ, having redeemed us by his blood, also renews us by his Spirit; \Children{so that our lives may show love and gratitude to God; }so that we may be assured of our faith by the fruits; \Children{and so that by our godly behavior others may be won to Christ.}

\Quote{1 Peter 2:9--12\index[1pe]{2:9--12}}{But you are a chosen people, a royal priesthood, a holy nation, God's special possession, that you may declare the praises of him who called you out of darkness into his wonderful light. Once you were not a people, but now you are the people of God; once you had not received mercy, but now you have received mercy. Dear friends, I urge you, as foreigners and exiles, to abstain from sinful desires, which wage war against your soul. Live such good lives among the pagans that, though they accuse you of doing wrong, they may see your good deeds and glorify God on the day he visits us.}

\section{Q 35. Since we are redeemed by grace alone, through faith alone, where does this faith come from?}
All the gifts we receive \Children{from} Christ we receive through \Children{the Holy Spirit}, including faith itself\Children{.}

\Quote{Titus 3:4--6\index[tit]{3:4--6}}{But when the kindness and love of God our Savior appeared, he saved us, not because of righteous things we had done, but because of his mercy. He saved us through the washing of rebirth and renewal by the Holy Spirit, whom he poured out on us generously through Jesus Christ our Savior.}

\chapter[Part 3. Spirit, Restoration, Growing In Grace][Spirit, Restoration, Growing In Grace]{Part 3. Spirit, Restoration, Growing In Grace}

\section{Q 36. What do we believe about the Holy Spirit?}
\Children{That he is God, coeternal with the Father and the Son}, and that God grants him irrevocably to all who believe\Children{.}

\Quote{John 14:16--17\index[joh]{14:16--17}}{I will ask the Father, and he will give you another advocate to help you and be with you forever\thinspace{}---\thinspace{}the Spirit of truth. The world cannot accept him, because it neither sees him nor knows him. But you know him, for he lives with you and will be in you.}

\section{Q 37. How does the Holy Spirit help us?}
\Children{The Holy Spirit convicts us of our sin}, comforts us, guides us, gives us spiritual gifts and the desire to obey God; \Children{and he enables us to pray and to understand God's Word.}

\Quote{Ephesians 6:17--18\index[eph]{6:17--18}}{Take the helmet of salvation and the sword of the Spirit, which is the word of God. And pray in the Spirit on all occasions with all kinds of prayers and requests.}

\section{Q 38. What is prayer?}
\Children{Prayer is pouring out our hearts to God} in praise, petition, confession of sin, and thanksgiving\Children{.}

\Quote{Psalm 62:8\index[psa]{62:8}}{Trust in him at all times, you people; pour out your hearts to him, for God is our refuge.}

\section{Q 39. With what attitude should we pray?}
\Children{With love, perseverance, and gratefulness}; in humble submission to God's will, knowing that, for the sake of Christ, he always hears our prayers\Children{.}

\Quote{Philippians 4:6\index[phi]{4:6}}{Do not be anxious about anything, but in every situation, by prayer and petition, with thanksgiving, present your requests to God.}

\section{Q 40. What should we pray?}
\Children{The whole Word of God directs }and inspires \Children{us in what we should pray}, including the prayer Jesus himself taught us\Children{.}

\Quote{Ephesians 3:14--21\index[eph]{3:14--21}}{For this reason I kneel before the Father, from whom every family in heaven and on earth derives its name. I pray that out of his glorious riches he may strengthen you with power through his Spirit in your inner being, so that Christ may dwell in your hearts through faith. And I pray that you, being rooted and established in love, may have power, together with all the Lord's holy people, to grasp how wide and long and high and deep is the love of Christ, and to know this love that surpasses knowledge\thinspace{}---\thinspace{}that you may be filled to the measure of all the fullness of God. Now to him who is able to do immeasurably more than all we ask or imagine, according to his power that is at work within us, to him be glory in the church and in Christ Jesus throughout all generations, for ever and ever! Amen.}

\section{Q 41. What is the Lord's Prayer?}
\Children{Our Father in heaven, hallowed\footnote{Editor's note: or ``holy''.} be your name, your kingdom come, your will be done, on earth as it is in heaven. Give us today our daily bread. And forgive us our debts, as we also have forgiven our debtors. And lead us not into temptation, but deliver us from evil.}

\Quote{Matthew 6:9\index[mat]{06:9@6:9}}{This, then, is how you should pray: ``Our Father in heaven, hallowed be your name {\dots}''}

\section{Q 42. How is the Word of God to be read and heard?}
\Children{With diligence, preparation, and prayer; so that we may accept it with faith}, store it in our hearts,\Children{\ and practice it in our lives.}

\Quote{2 Timothy 3:16--17\index[2ti]{3:16--17}}{All Scripture is God-breathed and is useful for teaching, rebuking, correcting and training in righteousness, so that the servant of God may be thoroughly equipped for every good work.}

\section{Q 43. What are the sacraments or ordinances?}
The sacraments or ordinances given by God and instituted by Christ, namely \Children{baptism and the Lord's Supper}, are visible signs and seals that we are bound together as a community of faith by his death and resurrection\Children{.} By our use of them the Holy Spirit more fully declares and seals the promises of the gospel to us.

\Quote{Romans 6:4\index[rom]{06:4@6:4}}{We were therefore buried with him through baptism into death in order that, just as Christ was raised from the dead through the glory of the Father, we too may live a new life.}

\Quote{Luke 22:19--20\index[luk]{22:19--20}}{And he took bread, gave thanks and broke it, and gave it to them, saying, ``This is my body given for you; do this in remembrance of me.'' In the same way, after the supper he took the cup, saying, ``This cup is the new covenant in my blood, which is poured out for you.''}

\section{Q 44. What is baptism?}
\Children{Baptism is the washing with water in the name of the Father, the Son, and the Holy Spirit}; it signifies and seals our adoption into Christ, our cleansing from sin, and our commitment to belong to the Lord and to his church\Children{.}

\Quote{Matthew 28:19\index[mat]{28:19}}{Therefore go and make disciples of all nations, baptizing them in the name of the Father and of the Son and of the Holy Spirit.}

\section{Q 45. Is baptism with water the washing away of sin itself?}
\Children{No, only the blood of Christ }and the renewal of the Holy Spirit \Children{can cleanse us from sin.}

\Quote{Luke 3:16\index[luk]{03:16@3:16}}{John answered them all, ``I baptize you with water. But one who is more powerful than I will come, the straps of whose sandals I am not worthy to untie. He will baptize you with the Holy Spirit and fire.''}

\section{Q 46. What is the Lord's Supper?}
\Children{Christ commanded all Christians to eat bread and to drink from the cup in thankful remembrance of him} and his death\Children{.} The Lord's Supper is a celebration of the presence of God in our midst; bringing us into communion with God and with one another; feeding and nourishing our souls. It also anticipates the day when we will eat and drink with Christ in his Father's kingdom.

\Quote{1 Corinthians 11:23--26\index[1co]{11:23--26}}{The Lord Jesus, on the night he was betrayed, took bread, and when he had given thanks, he broke it and said, ``This is my body, which is for you; do this in remembrance of me.'' In the same way, after supper he took the cup, saying, ``This cup is the new covenant in my blood; do this, whenever you drink it, in remembrance of me.'' For whenever you eat this bread and drink this cup, you proclaim the Lord's death until he comes.}

\section{Q 47. Does the Lord's Supper add anything to Christ's atoning work?}
\Children{No, Christ died once for all.} The Lord's Supper is a covenant meal celebrating Christ's atoning work; as it is also a means of strengthening our faith as we look to him, and a foretaste of the future feast. But those who take part with unrepentant hearts eat and drink judgment on themselves.

\Quote{1 Peter 3:18\index[1pe]{3:18}}{For Christ died for sins once for all, the righteous for the unrighteous, to bring you to God.}

\section{Q 48. What is the church?}
God chooses and preserves for himself \Children{a community elect\-ed for eternal life and united by faith, who love, follow, learn from, and worship God together.} God sends out this community to proclaim the gospel and prefigure Christ's kingdom by the quality of their life together and their love for one another.

\Quote{2 Thessalonians 2:13\index[2th]{2:13}}{But we ought always to thank God for you, brothers and sisters loved by the Lord, because God chose you as first-fruits to be saved through the sanctifying work of the Spirit and through belief in the truth.}

\section{Q 49. Where is Christ now?}
\Children{Christ rose bodily from the grave on the third day after his death and is seated at the right hand of the Father}, ruling his kingdom and interceding for us, until he returns to judge and renew the whole world\Children{.}

\Quote{Ephesians 1:19--21\index[eph]{1:19--21}}{That power is the same as the mighty strength he exerted when he raised Christ from the dead and seated him at his right hand in the heavenly realms, far above all rule and authority, power and dominion, and every name that is invoked, not only in the present age but also in the one to come.}

\section{Q 50. What does Christ's resurrection mean for us?}
\Children{Christ triumphed over sin and death} by being physically resurrected\Children{, so that all who trust in him are raised to new life in this world and to everlasting life in the world to come.} Just as we will one day be resurrected, so this world will one day be restored. But those who do not trust in Christ will be raised to everlasting death.

\Quote{1 Thessalonians 4:13--14\index[1th]{4:13--14}}{Brothers and sisters, we do not want you to be uninformed about those who sleep in death, so that you do not grieve like the rest of mankind, who have no hope. For we believe that Jesus died and rose again, and so we believe that God will bring with Jesus those who have fallen asleep in him.}

\section{Q 51. Of what advantage to us is Christ's ascension?}
\Children{Christ }physically ascended on our behalf, just as he came down to earth physically on our account, and he \Children{is now advocating for us in the presence of his Father}, preparing a place for us,\Children{\ and also sends us his Spirit.}

\Quote{Romans 8:34\index[rom]{08:34@8:34}}{Christ Jesus who died\thinspace{}---\thinspace{}more than that, who was raised to life\thinspace{}---\thinspace{}is at the right hand of God and is also interceding for us.}

\section{Q 52. What hope does everlasting life hold for us?}
It reminds us \Children{that }this present fallen world is not all there is; soon \Children{we will live with and enjoy God forever }in the new city, \Children{in the new heaven and the new earth, where we will be }fully and \Children{forever freed from all sin }and will inhabit renewed, resurrection bodies \Children{in a renewed, restored creation.}

\Quote{Revelation 21:1--4\index[rev]{21:1--4}}{Then I saw a new heaven and a new earth, for the first heaven and the first earth had passed away, and there was no longer any sea. I saw the Holy City, the new Jerusalem, coming down out of heaven from God, prepared as a bride beautifully dressed for her husband. And I heard a loud voice from the throne saying, ``Look! God's dwelling place is now among the people, and he will dwell with them. They will be his people, and God himself will be with them and be their God. He will wipe every tear from their eyes. There will be no more death or mourning or crying or pain, for the old order of things has passed away.''}

\cleardoublepage

\end{document}