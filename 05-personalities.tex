\documentclass[00-main.tex]{subfiles}

\begin{document}
	
\backmatter

\pagestyle{myPlainStyle}

\part{Addenda}

\chapter{Brief Biographies}

For all citations from or references to these writers, refer to the index ``People'', starting on page \pageref{ind:people}.

\begin{description}
	\item[Baxter, Richard (1615--1691):]\label{baxter} An English Puritan, Baxter served as a chaplain in the army of Oliver Cromwell and as a pastor in Kidderminster. When James II was overthrown, he was persecuted and imprisoned for 18 months. He continued to preach, writing at the time that: ``I preached as never sure to preach again, and as a dying man to dying men.'' As well as his theological works he was a poet and hymn-writer. He also wrote his own Family Catechism (from which the quote on p.~\pageref{baxter:q3} is taken).
	
	\item[Brooks, Thomas (1608--1680):]\label{brooks} An English Puritan preacher, Brooks studied at Cambridge University before becoming rector of a church in London. He was ejected from his post, but continued to work in London even during the Great Plague. He wrote over a dozen books, most of which are devotional in character, \emph{The Mute Christian Under the Smarting Rod\/} being the best known.
	
	\item[Bullinger, Heinrich (1504--1575):]\label{bullinger} A Swiss reformer, and the successor of Zwingli as head of the Zurich church, Bullinger wrote both theological and historical works comprising some 127 titles. There exist about 12,000 letters from and to Bullinger, the most extended correspondence preserved from Reformation times. He corresponded with Henry VIII, Edward VI, and Elizabeth I of England, Christian II of Denmark, and Frederick III Elector Palatine, among others.
	
	\item[Calvin, John (1509--1564):]\label{calvin} A theologian, administrator, and pastor, Cal\-vin was born in France into a strict Roman Catholic family. It was in Geneva however where Calvin worked most of his life and organized the Reformed church. He wrote the \emph{Institutes of the Christian Religion}, the \emph{Geneva Catechism}, as well as numerous commentaries on Scripture.
	
	\item[Wesley, John (1703--1791):]\label{wesley} An English preacher and theologian, Wesley is largely credited, along with his brother Charles, with founding the English Methodist movement. He traveled generally on horseback, preaching two or three times each day, and is said to have preached more than 40,000 sermons. He also was a noted hymn-writer.
	
\end{description}


\end{document}