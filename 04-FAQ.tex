%\documentclass[00-main.tex]{subfiles}

\begin{document}

\part{Frequently Asked Questions}

\chapter{FAQ---{\em New City Catechism}}

Most of your questions will be answered by the introduction to \emph{New City Catechism}. Please read that first.

\section[At What Age Is The Children's Catechism Aimed?][Catechism For What Ages?]{At What Age Is The Children's Catechism Aimed?}

\index{Children!Age appropriateness}This very much depends on your children and your way of using the catechism. Memorization can begin at an early age but if you want to use the Bible verses and prayers then 4\textsuperscript{th} to 5\textsuperscript{th} graders will get the most out of it. On the other hand, if your children are able to memorize and recite the Apostles' Creed\index{Apostles' Creed} (the longest catechism answer) then they should be able memorize the entire \emph{New City Catechism}\/ with ease.

\section[Why Is Some Of The Text In Color In The Answers?][Why Is Some Text In Color?]{Why Is Some Of The Text In Color In The Answers?}

\index{Answers!Children}\index{Answers!Adults}In the adult version the children's answer appears in \Children{color}\footnote{Editor's note: in the black \& white version, the text is \textbf{bolded}.} to differentiate it from the longer adult answer. \emph{New City Catechism}\/ is a joint adult and children's catechism. In other words, the same questions are asked of both children and adults, and the children's answer is always part of the adult answer. This means that as parents are teaching it to their children they are learning their answer to the question at the same time, albeit an abridged version. The adult answer is always an expanded version of the children's answer and so the \Children{colored} text shows the children's answer within the adult one.

\section[In What Order Should I Go Through The Verses, Commentaries, And So On?][Additional Resources]{In What Order Should I Go Through The Verses, Commentaries, And So On?}

Start by reading the Bible verse that accompanies each question and answer, and seeing how it applies and how the question and answer derive from it. Then read the text commentary, and then watch the video commentary. If you have access to either of the further reading books, read the recommended chapter(s). End your time in prayer, using the attached prayer as a starting point and for inspiration.

\section[How Do I Use {\em New City Catechism}?][Using the {\em Catechism}]{How Do I Use {\em New City Catechism}?}

\index{Catechesis!Personal}\emph{New City Catechism}\/ consists of 52 questions and answers so the easiest way to use it is to memorize one question and answer each week of the year. Because it is intended to be dialogical it is best to learn it with others, enabling you to drill one another on the answers not only one at a time but once you have learned 10 of them, then 20 of them, and so on. The Bible verse, written and filmed commentary, and prayer that are attached to each question and answer can be used as your devotion on a chosen day of the week to help you think through and meditate on the issues and applications that arise from the question and answer.

\section[How Do I Use {\em New City Catechism}\/ With My Family?][{\em New City Catechism}\/ With A Family]{How Do I Use {\em New City Catechism}\/ With My Family?}\index{Catechesis!Family}

\emph{New City Catechism}\/ consists of 52 questions and answers so the easiest way to use it is to memorize one question and answer together as a family each week of the year. It is intended for parents to help their children memorize the children's answer and then for parents to learn the longer, extended adult answer themselves. Parents will have different ways of approaching the memorization process depending on their children and their particular circumstances-so there are no prescribed times of day or particular devotional practices attached. When and how parents use the catechism can be as diverse as during family devotions, at the breakfast table, as part of a longer study including comprehension questions and praying, or as a fun memorization time with flashcards and drills. Parents may decide to read aloud the Bible verse and pray aloud the children's prayer attached to each question and answer, or it may be appropriate for your child to read and pray aloud themselves.

\section[How Do I Use {\em New City Catechism}\/ With My Study Group?][Catechism With Study Group]{How Do I Use {\em New City Catechism}\/ With My Study Group?}\index{Catechesis!Groups}

Groups may decide to spend the first 5--10 minutes of their study time looking together at only one question and answer thus completing the catechism in a year, or they may prefer to study and learn the questions and answers over a contracted length of time, for example by memorizing 5 or 6 questions a week and meeting together to quiz one another, discuss them, as well as read and watch the accompanying commentaries.

\section[Why Are Some Of The Prayers Longer Than Others?][Why Are Some Prayers Longer?]{Why Are Some Of The Prayers Longer Than Others?}\index{Prayers}

\index{Prayers}The prayers are intended to help and inspire you in prayer by showing you some of the ways historic preachers and authors prayed to and praised God. Please feel free to lengthen or shorten the prayers as is most helpful to you.\footnote{Editor's note: the prayers may be found at www.\hspace{0em}new\hspace{0em}city\hspace{0em}cate\hspace{0em}chism.\hspace{0em}com.}

\section[Which Catechism Should I Learn After This One?][Next Catechism?]{Which Catechism Should I Learn After This One?}

\emph{New City Catechism}\/ is based on and adapted from Calvin's\index[people]{Calvin, John} {\em Geneva Catechism}\index[books]{Geneva Catechism}, the {\em Westminster Shorter}\index[books]{Westminster Shorter Catechism}\/ and {\em Larger Catechism}s\index[books]{Westminster Larger Catechism}, and especially the {\em Heidelberg Catechism}\index[books]{Heidelberg Catechism}. A good next step would be to learn either Westminster Shorter or Heidelberg.

\section*{Any Additional Resources You Would Recommend?}

Kevin DeYoung %
\index[people]{DeYoung, Kevin}%
has written an excellent exploration of the \emph{Heidelberg Catechism\/}%
\index[books]{Heidelberg Catechism}%
in \emph{The Good News We Almost Forgot: Rediscovering the Gospel in a 16\textsuperscript{th} Century Catechism\/}%
\index[books]{Good News We Almost Forgot, The} %
(published by Moody).\nocite{DeYoung:2010}

Thomas Watson's\index[people]{Watson, Thomas} \emph{A Body of Divinity\/} (published by Banner of Truth, among others) is a great exposition of the \emph{Westminster Shorter Catechism}. %
\index[books]{Westminster Shorter Catechism}\nocite{Watson:1957}

Thomas F.\ Torrance's %
\index[people]{Torrance, Thomas F.} %
\emph{The School of Faith, Catechisms of the Reformed Church} (published by Wipf \& Stock) has a fascinating introduction to catechesis, as well as being a great collection of the historical catechisms.\nocite{Torrance:1996}

\emph{Grounded in the Gospel\/} by Gary Parrett %
\index[people]{Parrett, Gary} %
and J.~I.\ Packer %
\index[people]{Packer, J.I.} %
(published by Baker) provides a case for why catechetical instruction is still important for churches and discipleship today.\nocite{Gary-Parrett:2010}

\cleardoublepage

\end{document}