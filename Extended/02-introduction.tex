\documentclass[00-main.tex]{subfiles}

\begin{document}


\pagestyle{myPageStyle}
\pagenumbering{arabic}

\mainmatter

\chapter{Introduction}

\hspace{3ex}\textbf{Question 1. What is the chief end of man?\newline
Answer.} Man's chief end is to glorify God and to enjoy him forever.

\textbf{Question 1. What is your only comfort in life and death?\newline
Answer.} That I am not my own, but belong\thinspace{}---\thinspace{}body and soul, in life and in death\thinspace{}---\thinspace{}to my faithful Savior, Jesus Christ.

\smallskip

These words, the opening of the Westminster\index{Westminster Catechisms} and {\em Heidelberg Catechism}s,\index[book]{Heidelberg Catechism} find echoes in many of our creeds and statements of faith. They are familiar to us from sermons and books, and yet most people do not know their source and have certainly never memorized them as part of the catechisms from which they derive.

Today many churches and Christian organizations publish ``statements of faith''\index{Statements of faith} that outline their beliefs. But in the past it was expected that documents of this nature would be so biblically rich and carefully crafted that they would be memorized and used for Christian growth and training. They were written in the form of questions and answers\index{Questions \& answers}, and were called catechisms (from the Greek {\em katechein}\index{katechein@{\em katechein}}\/ which means ``to teach orally or to instruct by word of mouth''). The \index[book]{Heidelberg Catechism}{\em Heidelberg Catechism}\/ of 1563\index{1563} and \index[book]{Westminster Shorter Catechism}\index[book]{Westminster Larger Catechism}{\em Westminster Shorter and Larger Catechisms}\/ of \index{1648}1648 are among the best known, and they serve as the doctrinal standards of many churches in the world today.

\section{The Lost Practice Of Catechesis}
\index{Catechesis!Need}At present, the practice of catechesis, particularly among adults, has been almost completely lost. Modern discipleship\index{Discipleship} programs concentrate on practices such as Bible study,\index{Bible study} prayer,\index{Prayer} fellowship,\index{Fellowship} and evangelism\index{Evangelism} and can at times be superficial when it comes to doctrine.\index{Doctrine} In contrast, the classic catechisms take students through the Apostles' Creed,\index{Apostles' Creed} the Ten Commandments,\index{Ten Commandments} and the Lord's Prayer\index{Lord's Prayer}\thinspace{}---\thinspace{}a perfect balance of biblical theology,\index{Theology} practical ethics,\index{Ethics} and spiritual experience. Also, the catechetical discipline of memorization\index{Memorization} drives concepts deeper into the heart and naturally holds students more accountable to master the material than do typical discipleship courses. Finally, the practice of question\thinspace{}---\thinspace{}answer\index{Questions \& answers} recitation brings instructors and students into a naturally interactive,\index{Interactivity} dialogical process of learning.

In short, catechetical instruction is less individualistic and more communal. Parents\index{Parents} can catechize their children.\index{Children} Church leaders\index{Leaders} can catechize new members with shorter catechisms and new leaders with more extensive ones. Because of the richness of the material, catechetical questions and answers may be integrated into corporate worship\index{Corporate worship} itself, where the church as a body can confess their faith and respond to God with praise.

Because we have lost the practice of catechesis today: ``Superficial smatterings of truth, blurry notions about God and godliness, and thoughtlessness about the issues of living\thinspace{}---\thinspace{}career-wise, community-wise, family-wise, and church-wise\thinspace{}---\thinspace{}are all too often the marks of evangelical congregations today {\dots}'' (From \index[book]{Grounded in the Gospel}{\em Grounded in the Gospel: Building Believers the Old-Fashioned Way}, by Gary Parrett\index[people]{Parrett, Gary} and J.~I.\ Packer,\index[people]{Packer, J.I.} published by Baker, 2010.)\index{Catechesis|)}

\section{Why Write New Catechisms?}
There are many ancient, excellent, and time-tested catechisms. Why expend the effort to write new ones? In fact, some people might suspect the motives of anyone who would want to do so. However, most people today do not realize that it was once seen as normal, important, and necessary for churches to continually produce new catechisms for their own use. The original Anglican\index{Anglicans} \index[book]{Book of Common Prayer}{\em Book of Common Prayer}\/ included a catechism. The Lutheran\index{Lutherans} churches had Luther's \index[book]{Luther's Large Catechism@Luther's {\em Large Catechism}}{\em Large Catechism}\/ and \index[book]{Luther's Small Catechism@Luther's {\em Small Catechism}}{\em Small Catechism}\/ of \index{1529}1529. The early Scottish\index{Scottish churches} churches though they had \index[book]{Geneva Catechism}Calvin's {\em Geneva Catechism}\/ of \index{1541}1541, and the \index[book]{Heidelberg Catechism}{\em Heidelberg Catechism}\/ of \index{1563}1563, went on to produce and use \emph{Craig's Catechism}\index[book]{Craig's Catechism@Craig's {\em Catechism}}\/ of \index{1581}1581, \emph{Duncan's Latin Catechism}\index[book]{Latin Catechism}\/ of 1595\index{1595}, and \emph{The New Catechism}\index[book]{New Catechism, The}\/ of \index{1644}1644, before eventually adopted the \emph{Westminster Catechism}\index[book]{Westminster Catechisms}.

The Puritan pastor Richard Baxter\index[people]{Baxter, Richard}, who ministered in the 17\textsuperscript{th} century town of Kidderminster,\index{Kidderminster} was not unusual. He wanted to systematically train heads of families to instruct their households in the faith. To do so he wrote his own \emph{Family Catechism}\index[book]{Family Catechism}\/ that was adapted to the capacities of his people and that brought the Bible to bear on many of the issues and questions his people were facing at that time.

\index{Catechesis!Purposes|(}Catechisms were written with at least three purposes. The first was to set forth a comprehensive exposition of the gospel\thinspace{}---\thinspace{}not only in order to explain clearly what the gospel is, but also to layout the building blocks on which the gospel is based, such as the biblical doctrine of God, of human nature, of sin, and so forth. The second purpose was to do this exposition in such a way that the heresies, errors, and false beliefs of the time and culture were addressed and counteracted. The third and more pastoral purpose was to form a distinct people, a counter-culture that reflected the likeness of Christ not only in individual character but also in the church's communal life.

When looked at together, these three purposes explain why new catechisms must be written. While our exposition of gospel doctrine must be in line with older catechisms that are true to the Word, culture changes and so do the errors, temptations, and challenges to the unchanging gospel that people must be equipped to face and answer.\index{Catechesis!Purposes|)}

\section{A Joint Adult And Children's Catechism}
\emph{New City Catechism}\/ is comprised of only 52 questions and answers\index{Questions \& answers} (as opposed to \emph{Heidelberg's}\/ 129\index[book]{Heidelberg Catechism} or \index[book]{Westminster Shorter Catechism}\emph{Westminster Shorter's}\/ 107).\footnote{Editor's note: or the ACNA's\index{ACNA} 345 questions.} There is therefore only one question and answer for each week of the year, making it simple to fit into church calendars and achievable even for people with demanding schedules.

It is a joint adult and children's catechism. In other words, the same questions are asked of both children and adults, and the children's answer\index{Answers!Children} is always part of the adult answer\index{Answers!Adults}. This means that as parents are teaching it to their children they are learning their answer to the question at the same time, albeit an abridged version. The adult answer is always an expanded version of the children's answer. In the adult version the children's answer appears in \Children{color}\footnote{Editor's note: in the black \& white version, the text is \textbf{bolded}.} to differentiate it from the longer adult answer.

\emph{New City Catechism}\/ is based on and adapted from \index[book]{Geneva Catechism}\emph{Calvin's Geneva Catechism}, the \index[book]{Westminster Shorter Catechism}\emph{Westminster Shorter}\/ and \index[book]{Westminster Larger Catechism}\emph{Larger Catechism}s, and especially the \index[book]{Heidelberg Catechism}\emph{Heidelberg Catechism}, giving good exposure to some of the riches and insights across the spectrum of the great Reformation-era\index{Reformation} catechisms, the hope being that it will encourage people to delve into the historic catechisms and continue the catechetical process throughout their lives.

It is divided into 3 parts to make it easier to learn in sections and to include some helpful divisions:
\index{Catechism sections}
\begin{itemize}
	\item Part 1 = God, creation and fall, law (20 questions);
	\item Part 2 = Christ, redemption, grace (15 questions);
	\item Part 3 = Spirit, restoration, growing in grace (17 questions).
\end{itemize}

As with most traditional catechisms there is a Bible verse\index{Bible verses} that accompanies each question and answer. In addition, attached to each question and answer there is a short commentary and a short prayer\footnote{Editor's note: these prayers and commentaries can be found at http:/\slash{}www.new\hspace{0em}city\hspace{0em}cate\hspace{0em}chism.\hspace{0em}com.} taken from the writings or sayings of past preachers to help students meditate on and think about the topic being explored. As far as possible a commentary and prayer has been included from the same preacher in each of the 3 Parts so that students can become familiar with their style and work. Those quoted in all 3 Parts are, in chronological order: John Chrysostom\index[people]{Chrysostom, John}, Augustine of Hippo\index[people]{Augustine of Hippo}, Martin Luther\index[people]{Luther, Martin}, John Calvin\index[people]{Calvin, John}, Richard Sibbes\index[people]{Sibbes, Richard}, John Bunyan\index[people]{Bunyan, John}, Jonathan Edwards\index[people]{Edwards, Jonathan}, John Wesley\index[people]{Wesley, John}, Abraham Booth\index[people]{Booth, Abraham}, Charles Haddon Spurgeon\index[people]{Spurgeon, Charles Haddon}, John Charles Ryle\index[people]{Ryle, John Charles}, C.~S.\ Lewis\index[people]{Lewis, C.S.}, David Martyn Lloyd-Jones\index[people]{Lloyd-Jones, David Martyn}, and John Stott\index[people]{Stott, John}. Students are therefore able to read 3 commentaries and 3 prayers from each of these preachers. John Owen\index[people]{Owen, John} and Richard Baxter\index[people]{Baxter, Richard} have been quoted in Parts 1 and 3. John Bradford\index[people]{Bradford, John}, Heinrich Bullinger\index[people]{Bullinger, Heinrich}, Thomas Brooks\index[people]{Brooks, Thomas}, George Whitefield\index[people]{Whitefield, George}, Charles Simeon\index[people]{Simeon, Charles}, and Francis Schaeffer\index[people]{Schaeffer, Francis} feature once with a commentary and a prayer from each.

In the children's version the questions and answers are accompanied by the same Bible verse\index{Bible verses} as the adult version. In addition the prayers\index{Prayers} from the adult version have been adapted, modernized, shortened, and simplified for children.

Also included in the adult version is a further reading\index{Further reading}\footnote{Editor's note: remember that all additional elements of the \emph{New City Catechism}\/ not included in this document are found on the web site: http:/\slash{}www.new\hspace{0em}city\hspace{0em}cate\hspace{0em}chism.\hspace{0em}com.} section. In order to make this as manageable as possible suggested readings are drawn from only two books: J.~I.\ Packer's\index[people]{Packer, J.I.} \emph{Concise Theology}\index[book]{Concise Theology}\/ (published by Tyndale) and Donald Macleod's\index[people]{Macleod, Donald} \emph{A Faith to Live By}\index[book]{A Faith to Live By}\/ (published by Mentor or Christian Focus).

To accompany all this written material there are also short video commentaries from some of the council members of The Gospel Coalition\index{Gospel Coalition, The} and the pastors of Redeemer Presbyterian Church\index{Redeemer Presbyterian Church}. As with the textual commentaries from historic preachers, as far as possible, a video commentary from each of the current preachers has been included in each of the 3 Parts. Those featured in the filmed commentaries are, in alphabetical order: Thabiti Anyabwile\index[people]{Anyabwile, Thabiti}, Alistair Begg\index[people]{Begg, Alistair}, David Bisgrove\index[people]{Bisgrove, David}, D.~A.\ Carson\index[people]{Carson, D.A.}, Mark Dever\index[people]{Dever, Mark}, Kevin DeYoung\index[people]{DeYoung, Kevin}, Ligon Duncan\index[people]{Duncan, Ligon}, Joshua Harris\index[people]{Harris, Joshua}, Kent Hughes\index[people]{Hughes, Kent}, Timothy Keller\index[people]{Keller, Timothy}, John Lin\index[people]{Lin, John}, Crawford Loritts\index[people]{Loritts, Crawford}, John Piper\index[people]{Piper, John}, Juan Sanchez\index[people]{Sanchez, Juan}, Leo Schuster\index[people]{Schuster, Leo}, Stephen Um\index[people]{Um, Stephen}, and John Yates\index[people]{Yates, John}. The hope is that the textual and filmed commentaries provide complementary insights into the theme of each particular question and answer.

\section{The Use Of Archaic Language}
\index{Archaic Language, Use of|(}Although it may make the content seem less accessible at first glance, the language of the original texts has been retained as much as possible throughout the commentaries and prayers.

When people complained to J.R.R.\ Tolkien\index[people]{Tolkien, J.R.R.} about the archaic language he sometimes used, he answered that language carries cultural values and therefore his use of older forms was not nostalgia\thinspace{}---\thinspace{}it was principled. He believed that older ways of speaking conveyed older ways of understanding life that modern forms cannot convey, because modern language is enmeshed with modern views of life.

As an example, Tolkien points to a passage in \emph{The Lord of the Rings}\index[book]{Lord of the Rings, The}\/ where members of the Fellowship are choosing weapons and the (archaic) wording runs as follows: ``Helms too they chose.'' Some (wrongly) class the wording as an ``inversion'', since normal order is ``They also chose helmets'' or ``They chose helmets too.'' But, Tolkien comments that modern English has lost the trick of putting the word that one desires to be emphasized (for pictorial, emotional, or logical reasons) into prominent first place, without the addition of a lot of little ``empty'' words. The much terser and more vivid ancient styles often convey gravity and meaning in a way they would not were they modernized. (See Tolkien's letter to Professor Hugh Brogan\index[people]{Brogan, Hugh, Professor} in \emph{The Letters of J.R.R.\ Tolkien}\index[book]{Letters of J.R.R.~Tolkien, The}, edited by Humphrey Carpenter, published by Houghton Mifflin, 1981.)

For this reason, except in cases where the words are no longer in common use and are therefore incomprehensible (in which instances they often have been replaced with ellipses) the language and spelling of the original authors has been retained throughout the commentaries and prayers. Occasionally this is also true in some of the questions and answers where the more poetic forms\index{Poetic forms} aid memorization.\index{Archaic Language, Use of|)}


\section{How To Use \emph{New City Catechism}}
\index{Catechesis!Usage|(}\emph{New City Catechism}\/ consists of 52 questions and answers so the easiest way to use it is to memorize one question and answer each week of the year. Because it is intended to be dialogical it is best to learn it in pairs, in families, or as study groups, enabling you to drill one another on the answers not only one at a time but once you have learned 10 of them, then 20 of them, and so on.

The Bible verse, written and filmed commentary, and prayer that are attached to each question and answer can be used as your devotion on a chosen day of the week to help you think through and meditate on the issues and applications that arise from the question and answer. Note that some of the prayers are not directly addressed to God but are more exhortational in nature. As you read these prayers you can make them your own by praying the petitions to God or by taking the statements and turning them into petitions and prayers. For example if the text says: ``I love the Lord for he heard my voice and heard my cry for mercy.'' You can pray: ``Lord, I love you because so many times, you have heard my voice and my cry for mercy.''

Groups may decide to spend the first 5--10 minutes of their study time looking together at only one question and answer thus completing the catechism in a year, or they may prefer to study and learn the questions and answers over a contracted length of time, for example by memorizing 5 or 6 questions a week and meeting together to quiz one another, discuss them, as well as read and watch the accompanying commentaries.

For families, it is intended for parents to help their children memorize the children's answer and then for parents to learn the longer, extended adult answer themselves. Parents will have different ways of approaching the memorization process depending on their children and their particular circumstances\thinspace{}---\thinspace{}so there are no prescribed times of day or particular devotional practices attached. When and how parents use the catechism can be as diverse as during family devotions, at the breakfast table, as part of a longer study including comprehension questions and praying, or as a fun memorization time with flashcards and drills.

\section{Memorization Tips}
\index{Catechesis!Memorization tips|(}There are a variety of ways to commit texts to memory and some techniques suit certain learning styles better than others. A few examples include:

\begin{itemize}
	\item Read the question and answer out loud, and repeat, repeat, repeat.
	\item Read the question and answer out loud, try to repeat them without looking. Repeat.
	\item Read aloud through all Part 1 questions and answers (then 2, then 3) while moving about. The combination of movement and speech strengthens a person's ability to recall text.
	\item Record yourself saying all Part 1 questions and answers (then 2, then 3) and listen to them during everyday activities e.g. work-outs, chores, etc.
	\item Write the questions and answers on cards and tape them in a conspicuous area. Read them aloud every time you see them.
	\item Make flashcards with the question on one side and the answer on the other, and test yourself. Children can color these in and draw pictures on them.
	\item Review the question and answer at night and in the morning. For children spend a few minutes at bedtime helping them remember the answer, then repeat at breakfast the next morning.
	\item Write out the question and answer. Repeat. The process of writing also helps a person's ability to recall text.
	\item Drill the questions and answers with another person as often as possible.
\end{itemize} \index{Catechesis!Memorization tips|)} \index{Catechesis!Usage|)}

\section{A Biblical Practice}
In his letter to the Galatians Paul writes, ``Anyone who receives instruction in the word must share all good things with his instructor'' (Galatians 6:6)\index[gal]{6:6}. The Greek word for ``anyone who receives instruction'' is the word \emph{kate\-chou\-men\-os}\index{katechoumenos@{\em katechoumenos}}, one who is catechized. In other words, Paul is talking about a body of Christian doctrine ``catechism'') that was taught to them by an instructor (here the word ``catechizer''). The words ``all good things'' probably means financial support as well. In this light, the word \emph{koinoneo}\index{koinoneo@{\em koinoneo}}\thinspace{}---\thinspace{}which means ``to share'' or ``to have fellowship''\thinspace{}---\thinspace{}becomes even richer. The salary of a Christian teacher is not to be seen simply as a payment but a ``fellowship.'' Catechesis is not just one more service to be paid for, but is a rich fellowship and mutual sharing of the gifts of God.

If we re-engage in this biblical practice in our churches, we will find again God's Word ``dwelling in us richly'' (Colossians 3:16)\index[col]{3:16}, because the practice of catechesis takes truth deep into our hearts, so we find ourselves thinking in biblical categories as soon as we can reason.

When my son, Jonathan, was a young child my wife Kathy and I started teaching him a children's catechism. In the beginning we worked on just the first three questions:

\textbf{Question 1. Who made you?\newline
Answer.} God.

\textbf{Question 2. What else did God make?\newline
Answer.} God made all things.

\textbf{Question 3. Why did God make you and all things?\newline
Answer.}\ \ For his own glory.

One day Kathy dropped Jonathan off at a babysitter's. At one point the babysitter discovered Jonathan looking out the window. ``What are you thinking about?'' she asked him. ``God,'' he said. Surprised, she responded, ``What are you thinking about God?'' He looked at her and replied, ``How he made all things for his own glory.'' She thought she had a spiritual giant on her hands! A little boy looking out the window, contemplating the glory of God in creation!

What had actually happened, obviously, was that her question had triggered the question/answer response in him. He answered with the catechism. He certainly did not have the slightest idea what the ``glory of God'' meant. But the concept was in his mind and heart, waiting to be connected with new insights, teaching, and experiences.

Such instruction, Princeton theologian Archibald Alexander\index[people]{Alexander, Archibald} said, is like firewood in a fireplace. Without the fire\thinspace{}---\thinspace{}the Spirit of God\thinspace{}---\thinspace{}firewood will not in itself produce a warming flame. But without fuel there can be no fire either, and that is what catechetical instruction is.

\bigskip

{\raggedleft
Timothy Keller, October 2012}

\cleardoublepage

\end{document}
